\chapter{Recursos LaTeX}

\begin{center}
    \textit{En que consiste este capítulo (opcional).}
\end{center}

Explicación e introducción al capítulo. 

Blah blah blah blah blah blah blah blah blah blah blah
blah blah blah blah blah blah blah blah blah blah blah 
blah blah blah blah blah blah blah blah blah blah blah 
blah blah blah blah blah blah.

Blah blah blah blah blah blah blah blah blah blah blah 
blah blah blah blah blah blah.

\section{Fuente}

\subsection{Familia}
{\ttfamily typewriter (máquina de escribir)}\\
{\sffamily sans serif}\\
{\rmfamily roman}

\subsection{Forma}
\textbf{texto en negritas}\\
\textit{texto en itálicas}\\
\textsl{texto inclinado}\\
\texttt{texto en estilo máquina de escribir}\\
\textsc{texto en mayúsculas pequeñas}

\subsection{Tamaño}
{\tiny  tiny - texto de prueba}\\
{\scriptsize  scriptsize - texto de prueba}\\
{\footnotesize  footnotesize - texto de prueba}\\
{\small  small - texto de prueba}\\
{\normalsize  normalsize - texto de prueba}\\
{\large  large - texto de prueba}\\
{\Large  Large - texto de prueba}\\
{\LARGE  LARGE - texto de prueba}\\
{\huge  huge - texto de prueba}\\
{\Huge  Huge - texto de prueba}

\section{Listado}

\subsection{No numerados}
\begin{itemize}
    \item Item 1
    \item Item 2
    \item Item 3
\end{itemize}

\subsection{Numerados}
\begin{enumerate}
    \item Item 1
    \item Item 2
    \item Item 3
\end{enumerate}


\section{Referenciación con APA}

\subsection{Citación como parte de párrafo}

\subsection*{Un autor}

Como menciona \citeA{libro:ejemplo}, no es la única
forma de citar.

\subsection*{Varios autores}

Como menciona \citeA{libro:ejemplo_varios_autores}, no es la única
forma de citar.

\subsection{Citación en la parte final}

\subsection*{Un autor}

Duis fringilla tristique neque. Sed interdum libero ut metus.
Pellentesque placerat. Nam rutrum augue a leo. Morbi sed 
elit sit amet ante lobortis sollicitudin.Praesent blandit 
blandit mauris citumoris totalis. \cite{libro:ejemplo}.

\subsection*{Varios autores}

Duis fringilla tristique neque. Sed interdum libero ut metus.
Pellentesque placerat. Nam rutrum augue a leo. Morbi sed 
elit sit amet ante lobortis sollicitudin.Praesent blandit 
blandit mauris citumoris totalis. \cite{libro:ejemplo_varios_autores}.

\subsection{Citación con número de página}

Como menciona \citeA[p.~5]{libro:ejemplo}, no es la única
forma de citar.

Duis fringilla tristique neque. Sed interdum libero ut metus.
Pellentesque placerat. Nam rutrum augue a leo. Morbi sed 
elit sit amet ante lobortis sollicitudin.Praesent blandit 
blandit mauris citumoris totalis. 
\cite[p.~7-12]{libro:ejemplo_varios_autores}.

\subsection{Citación anexos}
Duis fringilla tristique neque. Sed interdum libero ut metus.
Pellentesque placerat (Ver Anexo A).

\section{Figuras}
Referenciando a la figura \ref{fig:ejemplo}.

\begin{figure}[!h]
    \centering
    \includegraphics[width=8cm]{img/capitulo_1/figura_ejemplo.png}
    \caption{Explicación de la figura (Aquí)}
    Fuente: Adaptada de Apellido, N. (2000) \textit{Nombre del libro}.
    Editorial o universidad que lo publicó.
    \label{fig:ejemplo}
\end{figure}

\section{Tablas}
Sección en la que se detallan el uso de tablas.

\subsection{Corto}
Con respecto a la tabla \ref{tabla:ejemplo}, se tiene la siguiente 
información.

\begin{table}[!h]
    \begin{center}
        \caption{Título de la tabla}
        \label{tabla:ejemplo}
        \begin{tabular}{c|c|c|c|}
            \cline{2-4}
            & \textbf{Columna 1} & \textbf{Columna 2} & \textbf{Columna 3} \\ \hline
            \multicolumn{1}{|c|}{\textbf{Fila 1}} & item               & item               & item               \\ \hline
            \multicolumn{1}{|c|}{\textbf{Fila 2}} & item               & item               & item               \\ \hline
            \multicolumn{1}{|c|}{\textbf{Fila 3}} & item               & item               & item               \\ \hline
        \end{tabular}
    \end{center}
    \vspace{4mm}
    Nota. Extraída de Apellido, N. (2000) \textit{Nombre del libro}.
    Editorial o universidad que lo publicó.
\end{table}

\subsection{Multipágina}
Stique neque. Sed interdum libero ut metus.
Pellentesque placerat. Nam rutrum augue a leo. Morbi sed 
elit sit amet ante lobortis sollicitudin.Praesent blandit 
blandit mauris. Praesent lectus tellus, aliquet aliquam,
luctus a, egestas a, turpis.Mauris lacinia lorem sit amet
ipsum. Nunc quis urna dictum turpis accumsan semper. Tabla
\ref{tabla:tabla_largo_ejemplo}.

\begin{longtable}{p{2cm}|p{3cm}|p{3cm}|p{3cm}|p{3cm}|}
    \caption{Titulo de tabla multipágina} \label{tabla:tabla_largo_ejemplo} \\
    \cline{2-5}
    \multicolumn{1}{l|}{} & \multicolumn{1}{c|}{\textbf{Columna 1}} & \multicolumn{1}{c|}{\textbf{Columna 2}} & \multicolumn{1}{c|}{\textbf{Columna 3}} & \multicolumn{1}{c|}{\textbf{Columna 4}}\\ \hline
    \endfirsthead

    \multicolumn{5}{c}{{\tablename{} \thetable{} -- Continuación de tabla previa}} \\
    \multicolumn{5}{l}{} \\
    \cline{2-5}
    \multicolumn{1}{l|}{} & \multicolumn{1}{c|}{\textbf{Columna 1}} & \multicolumn{1}{c|}{\textbf{Columna 2}} & \multicolumn{1}{c|}{\textbf{Columna 3}} & \multicolumn{1}{c|}{\textbf{Columna 4}}\\ \hline
    \endhead

    \multicolumn{5}{c}{{Continua en la siguiente página.}} \\
    \endfoot

    \multicolumn{5}{l}{} \\
    \multicolumn{5}{l}{Nota. Extraída de Apellido, N. (2000) \textit{Nombre del libro}. Editorial o universidad que lo publicó.} \\
    \endlastfoot

    \multicolumn{1}{|c|}{\textbf{Fila 1}} & Lorem ipsum dolor sit amet, consectetuer adipiscing elit. & Lorem ipsum dolor sit amet, consectetuer adipiscing elit. & Lorem ipsum dolor sit amet, consectetuer adipiscing elit. & Lorem ipsum dolor sit amet, consectetuer adipiscing elit. \\ \hline
    \multicolumn{1}{|c|}{\textbf{Fila 2}} & Lorem ipsum dolor sit amet, consectetuer adipiscing elit. & Lorem ipsum dolor sit amet, consectetuer adipiscing elit. & Lorem ipsum dolor sit amet, consectetuer adipiscing elit. & Lorem ipsum dolor sit amet, consectetuer adipiscing elit. \\ \hline
    \multicolumn{1}{|c|}{\textbf{Fila 3}} & Lorem ipsum dolor sit amet, consectetuer adipiscing elit. & Lorem ipsum dolor sit amet, consectetuer adipiscing elit. & Lorem ipsum dolor sit amet, consectetuer adipiscing elit. & Lorem ipsum dolor sit amet, consectetuer adipiscing elit. \\ \hline
    \multicolumn{1}{|c|}{\textbf{Fila 4}} & Lorem ipsum dolor sit amet, consectetuer adipiscing elit. & Lorem ipsum dolor sit amet, consectetuer adipiscing elit. & Lorem ipsum dolor sit amet, consectetuer adipiscing elit. & Lorem ipsum dolor sit amet, consectetuer adipiscing elit. \\ \hline
    \multicolumn{1}{|c|}{\textbf{Fila 5}} & Lorem ipsum dolor sit amet, consectetuer adipiscing elit. & Lorem ipsum dolor sit amet, consectetuer adipiscing elit. & Lorem ipsum dolor sit amet, consectetuer adipiscing elit. & Lorem ipsum dolor sit amet, consectetuer adipiscing elit. \\ \hline
    \multicolumn{1}{|c|}{\textbf{Fila 6}} & Lorem ipsum dolor sit amet, consectetuer adipiscing elit. & Lorem ipsum dolor sit amet, consectetuer adipiscing elit. & Lorem ipsum dolor sit amet, consectetuer adipiscing elit. & Lorem ipsum dolor sit amet, consectetuer adipiscing elit. \\ \hline  
    \multicolumn{1}{|c|}{\textbf{Fila 7}} & Lorem ipsum dolor sit amet, consectetuer adipiscing elit. & Lorem ipsum dolor sit amet, consectetuer adipiscing elit. & Lorem ipsum dolor sit amet, consectetuer adipiscing elit. & Lorem ipsum dolor sit amet, consectetuer adipiscing elit. \\ \hline    
    \multicolumn{1}{|c|}{\textbf{Fila 8}} & Lorem ipsum dolor sit amet, consectetuer adipiscing elit. & Lorem ipsum dolor sit amet, consectetuer adipiscing elit. & Lorem ipsum dolor sit amet, consectetuer adipiscing elit. & Lorem ipsum dolor sit amet, consectetuer adipiscing elit. \\ \hline    
    \multicolumn{1}{|c|}{\textbf{Fila 9}} & Lorem ipsum dolor sit amet, consectetuer adipiscing elit. & Lorem ipsum dolor sit amet, consectetuer adipiscing elit. & Lorem ipsum dolor sit amet, consectetuer adipiscing elit. & Lorem ipsum dolor sit amet, consectetuer adipiscing elit. \\ \hline    
    \multicolumn{1}{|c|}{\textbf{Fila 10}} & Lorem ipsum dolor sit amet, consectetuer adipiscing elit. & Lorem ipsum dolor sit amet, consectetuer adipiscing elit. & Lorem ipsum dolor sit amet, consectetuer adipiscing elit. & Lorem ipsum dolor sit amet, consectetuer adipiscing elit. \\ \hline    
    \multicolumn{1}{|c|}{\textbf{Fila 11}} & Lorem ipsum dolor sit amet, consectetuer adipiscing elit. & Lorem ipsum dolor sit amet, consectetuer adipiscing elit. & Lorem ipsum dolor sit amet, consectetuer adipiscing elit. & Lorem ipsum dolor sit amet, consectetuer adipiscing elit. \\ \hline    
    \multicolumn{1}{|c|}{\textbf{Fila 12}} & Lorem ipsum dolor sit amet, consectetuer adipiscing elit. & Lorem ipsum dolor sit amet, consectetuer adipiscing elit. & Lorem ipsum dolor sit amet, consectetuer adipiscing elit. & Lorem ipsum dolor sit amet, consectetuer adipiscing elit. \\ \hline    
\end{longtable}

\section{Fórmulas matemáticas}

\section*{Simple}
Referenciando fórmula en parrafo, fórmula \ref{eq:euler}.

\begin{equation}
    e^{i\pi} + 1 = 0
    \label{eq:euler}
\end{equation}

\section*{Matrices}
Referenciando fórmula en parrafo, fórmula \ref{eq:matriz}.

\begin{equation}
    \left[
    \begin{matrix}
     a & b & c \\
     d & e & f \\
     g & h & i
    \end{matrix}
    \right]
    \label{eq:matriz}
\end{equation}

\section*{Límites}
Referenciando fórmula en parrafo, fórmula \ref{eq:limites}.

\begin{equation}
    \lim_{x\rightarrow\infty}\frac{3+x}{x^2} 
    \label{eq:limites}
\end{equation}

\section{Diagramas de flujo}
A pesar de que en {\LaTeX} es posible realizar geniales diagramas de flujo,
a la larga es complicado mantenerlos, en ese sentido,
es recomendable utilizar alternativas como \url{https://app.diagrams.net/}
o mas conocido como draw.io, tiene instaladores para todos los sistemas 
operativos. 

\begin{center}
    \begin{tikzpicture}[node distance = 3cm, auto]
        \footnotesize
        % Nodos
        \node [block] (init) {Inicio};
        \node [block, below of=init] (test) {Nodo A};
        \node [block, right of=test] (search) {Nodo B};
        \node [decision, below of=test] (test_done) {Decisión A};
        \node [block, below of=test_done] (intregate) {Nodo C};
        \node [cloud, left of=intregate] (code) {Extra};
        \node [decision, below of=intregate] (all_done) {Decisión B};
        \node [block, below of=all_done] (stop) {Nodo D};
        % Cordenadas
        \coordinate[left of=test] (il);
        \coordinate[left of=all_done] (al);
        % Líneas
        \path [line, dashed] (code) -- (intregate);
        \path [line, dashed] (test_done) -| node [near start] {si} (code);
        \path [line] (init) -- (test);
        \path [line] (test) -- (test_done);
        \path [line] (test_done) -- node {si}(intregate);
        \path [line] (test_done) -| node  [near start] {no} (search);
        \path [line] (intregate) -- (all_done);
        \path [line] (all_done.west) |- node [near end] {no} ([xshift=-2.3cm]al) -- ([xshift=-2.3cm]il)-- (test.west);
        \path [line] (all_done) -- node {si} (stop);
        \path [line] (search) -- (test);
    \end{tikzpicture}
\end{center}

\section{Unidades}

Peso \si{69 \kilogram}.
Mido \si{2 \metre}.
Camino \si{5 \second} por día.
Mi calle tiene \ang{10} de diferencia con otra.
\ang{10;8;2}.

Para más unidades consultar la documentación del paquete 
\textbf{siunitx}.

\section{Código}
Ejemplo de referencia y uso de \ref{code:hello_world}.

\subsection{Código corto}
Para código corto puedes emplear a minted como un listing, código \ref{code:example_i2c}.
para cambiar el fondo o retiralo basta con ver la opción ``bgcolor''.

\begin{listing}[H]
    \caption{Función genérica de lectura I2C con ``Wire.h''.}
    \label{code:example_i2c}
    \begin{minted}[bgcolor=gray_background, xleftmargin=10pt, linenos=true, breaklines, breakanywhere, fontsize=\footnotesize]{cpp}
void i2c_read(uint8_t main_address, uint8_t address, uint8_t *buffer, size_t size) {
    Wire.beginTransmission(main_address);
    Wire.write(address);
    Wire.endTransmission();
    uint8_t segments = Wire.requestFrom(main_address, size);
    if (segments == (uint8_t) size) {
        for (uint8_t i = 0; i < size; i++)
        {
        buffer[i] = Wire.read();
        }  
    }
}
    \end{minted}
\end{listing}

\subsection{Código que excede una plana}

Para código enormes usa este ejemplo de código \ref{code:request_partial}.
\begin{code}
    \begin{minted}[bgcolor=gray_background, xleftmargin=10pt, linenos=true, breaklines, breakanywhere, fontsize=\footnotesize]{python}
"""
requests.utils
~~~~~~~~~~~~~~
This module provides utility functions that are used within Requests
that are also useful for external consumption.
"""

import codecs
import contextlib
import io
import os
import re
import socket
import struct
import sys
import tempfile
import warnings
import zipfile
from collections import OrderedDict

from urllib3.util import make_headers, parse_url

from . import certs
from .__version__ import __version__

# to_native_string is unused here, but imported here for backwards compatibility
from ._internal_utils import to_native_string  # noqa: F401
from .compat import (
    Mapping,
    basestring,
    bytes,
    getproxies,
    getproxies_environment,
    integer_types,
)
from .compat import parse_http_list as _parse_list_header
from .compat import (
    proxy_bypass,
    proxy_bypass_environment,
    quote,
    str,
    unquote,
    urlparse,
    urlunparse,
)
from .cookies import cookiejar_from_dict
from .exceptions import (
    FileModeWarning,
    InvalidHeader,
    InvalidURL,
    UnrewindableBodyError,
)
from .structures import CaseInsensitiveDict

NETRC_FILES = (".netrc", "_netrc")

DEFAULT_CA_BUNDLE_PATH = certs.where()

DEFAULT_PORTS = {"http": 80, "https": 443}

# Ensure that ', ' is used to preserve previous delimiter behavior.
DEFAULT_ACCEPT_ENCODING = ", ".join(
    re.split(r",\s*", make_headers(accept_encoding=True)["accept-encoding"])
)


if sys.platform == "win32":
    # provide a proxy_bypass version on Windows without DNS lookups

    def proxy_bypass_registry(host):
        try:
            import winreg
        except ImportError:
            return False

        try:
            internetSettings = winreg.OpenKey(
                winreg.HKEY_CURRENT_USER,
                r"Software\Microsoft\Windows\CurrentVersion\Internet Settings",
            )
            # ProxyEnable could be REG_SZ or REG_DWORD, normalizing it
            proxyEnable = int(winreg.QueryValueEx(internetSettings, "ProxyEnable")[0])
            # ProxyOverride is almost always a string
            proxyOverride = winreg.QueryValueEx(internetSettings, "ProxyOverride")[0]
        except OSError:
            return False
        if not proxyEnable or not proxyOverride:
            return False

        # make a check value list from the registry entry: replace the
        # '<local>' string by the localhost entry and the corresponding
        # canonical entry.
        proxyOverride = proxyOverride.split(";")
        # now check if we match one of the registry values.
        for test in proxyOverride:
            if test == "<local>":
                if "." not in host:
                    return True
            test = test.replace(".", r"\.")  # mask dots
            test = test.replace("*", r".*")  # change glob sequence
            test = test.replace("?", r".")  # change glob char
            if re.match(test, host, re.I):
                return True
        return False

    def proxy_bypass(host):  # noqa
        """Return True, if the host should be bypassed.
        Checks proxy settings gathered from the environment, if specified,
        or the registry.
        """
        if getproxies_environment():
            return proxy_bypass_environment(host)
        else:
            return proxy_bypass_registry(host)


def dict_to_sequence(d):
    """Returns an internal sequence dictionary update."""

    if hasattr(d, "items"):
        d = d.items()

    return d


def super_len(o):
    total_length = None
    current_position = 0

    if hasattr(o, "__len__"):
        total_length = len(o)

    elif hasattr(o, "len"):
        total_length = o.len

    elif hasattr(o, "fileno"):
        try:
            fileno = o.fileno()
        except (io.UnsupportedOperation, AttributeError):
            # AttributeError is a surprising exception, seeing as how we've just checked
            # that `hasattr(o, 'fileno')`.  It happens for objects obtained via
            # `Tarfile.extractfile()`, per issue 5229.
            pass
        else:
            total_length = os.fstat(fileno).st_size

            # Having used fstat to determine the file length, we need to
            # confirm that this file was opened up in binary mode.
            if "b" not in o.mode:
                warnings.warn(
                    (
                        "Requests has determined the content-length for this "
                        "request using the binary size of the file: however, the "
                        "file has been opened in text mode (i.e. without the 'b' "
                        "flag in the mode). This may lead to an incorrect "
                        "content-length. In Requests 3.0, support will be removed "
                        "for files in text mode."
                    ),
                    FileModeWarning,
                )

    if hasattr(o, "tell"):
        try:
            current_position = o.tell()
        except OSError:
            # This can happen in some weird situations, such as when the file
            # is actually a special file descriptor like stdin. In this
            # instance, we don't know what the length is, so set it to zero and
            # let requests chunk it instead.
            if total_length is not None:
                current_position = total_length
        else:
            if hasattr(o, "seek") and total_length is None:
                # StringIO and BytesIO have seek but no usable fileno
                try:
                    # seek to end of file
                    o.seek(0, 2)
                    total_length = o.tell()

                    # seek back to current position to support
                    # partially read file-like objects
                    o.seek(current_position or 0)
                except OSError:
                    total_length = 0

    if total_length is None:
        total_length = 0

    return max(0, total_length - current_position)
    \end{minted}
    \caption{Este es un ejemplo de hola mundo en Python.}
    \label{code:request_partial}
\end{code}

Ten mucho cuidado, en ocasiones caracteres especiales como tildes,
asteriscos, arrobas y demás se ven en la imposibilidad de no 
compilarse.

\subsection{Código desde un archivo externo}

Ejemplo de minted a través de un archivo de código externo, código \ref{code:demo_py}.

\begin{code}
    \inputminted[bgcolor=gray_background, xleftmargin=10pt, linenos=true, breaklines, fontsize=\footnotesize]{python}{codigo/capitulo_1/demo.py}
    \caption{Este es un ejemplo de hola mundo en Python.}
    \label{code:demo_py}
\end{code}

Tu código debe ir en la carpeta ``codigo'', puedes separarlo 
por capitulos en subcarpetas para mayor orden.