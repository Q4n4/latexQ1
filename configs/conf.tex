\renewcommand{\BOthers}[1]{et al.\hbox{}}
\renewcommand{\BAvailFrom}[1]{Recuperado de\hbox{}}
\renewcommand{\BRetrievedFrom}[1]{Recuperado de }
\renewcommand{\BRetrieved}[1]{}

% Anexos Letras
\renewcommand{\appendixname}{Anexo}
\renewcommand{\appendixtocname}{Anexos}
\renewcommand{\appendixpagename}{Anexos}

% Indices renombrados
\renewcommand{\listfigurename}{Índice de Figuras}
\renewcommand{\listtablename}{Índice de Tablas}

% References renombrado
\renewcommand{\bibname}{Bibliografía}

\renewcommand{\listoflistingscaption}{Índice de Códigos}
\renewcommand{\listingscaption}{\sffamily{Código}}

% Configuración de las formas en la graficación de diagramas de flujo
\tikzstyle{decision} = [diamond, draw, fill=blue!15, 
  text width=4.5em, text badly centered, node distance=3cm, inner sep=0pt]
\tikzstyle{block} = [rectangle, draw, fill=blue!15, 
  text width=5em, text centered, rounded corners, minimum height=4em]
\tikzstyle{line} = [draw, -latex']
\tikzstyle{cloud} = [draw, ellipse,fill=red!15, node distance=3cm,
  minimum height=2em]

% Configurando parskip
\setlength{\parindent}{0cm}
\setlength{\parskip}{5mm}

% Configurando caption
\captionsetup[table]{skip=4mm}

% Estilo de orientación, tamaño y font family de chapter title
\ChTitleVar{\huge\bfseries\sffamily\raggedleft}

% Configurando el espaciado entre "itemize/enumerate items"
\setlist[itemize]{noitemsep, topsep=3mm}
\setlist[enumerate]{noitemsep, topsep=3mm}

% Tamaño y espaciado de section/subsection/subsubsection/paragraph
\titleformat*{\section}{\Large\bfseries}
\titleformat*{\subsection}{\large\bfseries}
\titleformat*{\subsubsection}{\large\bfseries}
\titleformat{\paragraph}{\normalsize\bfseries}{\theparagraph}{1em}{}
\titlespacing*{\paragraph}
{0pt}{3.25ex plus 1ex minus .2ex}{1.5ex plus .2ex}

% Definiendo colores y estilos para minted
% otros estilos: manni, rrt, perldoc, colorful, borland, murphy, vs, trac
\usemintedstyle{tango}
\definecolor{gray_background}{gray}{0.92}

% Para agregar Bookmarks de imágenes y tablas
\makeatletter
  \newcommand{\listoffiguresbookmarks}{%
    \pdfbookmark[0]{\listfigurename}{listoffiguresbookmark}
    \bookmarksetup{level=1}
    \@starttoc{lofb}
  }
  \newcommand{\listoftablesbookmarks}{%
    \pdfbookmark[0]{\listtablename}{listoftablesbookmark}
    \bookmarksetup{level=1}
    \@starttoc{lotb}
  }
\makeatother

\makeatletter
  \pretocmd\endfigure{%
  \addtocontents{lofb}{%
    \protect{%
      \bookmark[
      rellevel=1,
      keeplevel,
      dest=\@currentHref,
      ]{Figura \thefigure: \@currentlabelname}}}%
  }{}{\errmessage{Patching \noexpand\endfigure failed}}
  
  \pretocmd\endtable{%
  \addtocontents{lotb}{%
    \protect{%
      \bookmark[
      rellevel=1,
      keeplevel,
      dest=\@currentHref,
      ]{Tabla \thetable: \@currentlabelname}}}%
  }{}{\errmessage{Patching \noexpand\endtable failed}}
\makeatother

\AtEndDocument{%
  \listoffiguresbookmarks
  \listoftablesbookmarks
}


% Environment para código largos 
\newenvironment{code}{\captionsetup{type=listing}}{
    {}    
}
\BeforeBeginEnvironment{code}{\vspace{2mm}}
% \AfterEndEnvironment{code}{\vspace{0.5mm}}


% Espacios entre autores en "Bibliografia"
\let\OLDthebibliography\thebibliography
\renewcommand\thebibliography[1]{
  \OLDthebibliography{#1}
  \setlength{\parskip}{3mm}
  \setlength{\itemsep}{3mm plus 0.3ex}
}