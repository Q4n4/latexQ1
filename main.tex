% Autor: Mauricio S. Matias Conde

% Clase de Documento, formato libro
\documentclass[12pt,letterpaper]{book}

% Paquete para el soporte de idiomas
\usepackage[spanish,es-tabla]{babel}

% Paquete para la manipulación de imágenes.
\usepackage{graphicx}

% Paquete para la referencias links y marcadores en el menu del pdf
\usepackage[bookmarksnumbered,bookmarksopen=false]{hyperref}
% Paquete para agregar "custom" bookmarks
\usepackage{bookmark}


% Márgenes de la hoja
\usepackage[top=2cm,bottom=2cm,left=3cm,right=2cm]{geometry}

% Paquete para encabezados o pies de página.
\usepackage{fancyhdr}

% Paquete para la prevensión de aparición 
% de pies de página, encabezados y numeración.
\usepackage{emptypage}

% Paquete para la adecuación de cuadros y tablas.
\usepackage{float}

% Paquete que permite la construcción de tablas
% que abarquen mas de una página.
\usepackage{longtable}

% Paquete que ofrece el soporte de citación 
% estilo APA.
\usepackage{apacite}

% Paquetes para el uso de fórmulas matemáticas.
\usepackage{amsmath, amsthm, amssymb}

% Paquete para la inclusión de Anexos
\usepackage{appendix}

% Paquete para la graficación
\usepackage{tikz}
\usetikzlibrary{shapes,arrows}

% Paquete para el espaciado automático entre párrafos
% ya no es necesario el "//"
\usepackage{parskip}

% Paquete para espaciar el titulo de las tablas 
\usepackage{caption}

% Paquete para eliminar los espacios en blanco 
% entre titulos/subtitulos/etc.
\usepackage[compact]{titlesec}

% Paquete para cambiar el estilo de 
% títulos de capítulos, secciones y 
% subsecciones
\usepackage[Lenny]{fncychap}

% Paquete pora importar de forma anidada
\usepackage{import}

% Paquete que simplifica la notación de unidades físicas medidas etc 
\usepackage{siunitx}

% Paquete para la configuración de listas
\usepackage{enumitem}

% Paquete "custom" para la incrustación de datos personales
\usepackage{configs/docstyle}

\begin{document}
    \renewcommand{\BOthers}[1]{et al.\hbox{}}
\renewcommand{\BAvailFrom}[1]{Recuperado de\hbox{}}
\renewcommand{\BRetrievedFrom}[1]{Recuperado de }
\renewcommand{\BRetrieved}[1]{}

% Anexos Letras
\renewcommand{\appendixname}{Anexos}
\renewcommand{\appendixtocname}{Anexos}
\renewcommand{\appendixpagename}{Anexos}

% Indices renombrados
\renewcommand{\listfigurename}{Índice de Figuras}
\renewcommand{\listtablename}{Índice de Tablas}

% References renombrado
\renewcommand{\bibname}{Bibliografía}

% Configuración de las formas en la graficación de diagramas de flujo
\tikzstyle{decision} = [diamond, draw, fill=blue!15, 
  text width=4.5em, text badly centered, node distance=3cm, inner sep=0pt]
\tikzstyle{block} = [rectangle, draw, fill=blue!15, 
  text width=5em, text centered, rounded corners, minimum height=4em]
\tikzstyle{line} = [draw, -latex']
\tikzstyle{cloud} = [draw, ellipse,fill=red!15, node distance=3cm,
  minimum height=2em]

% Configurando parskip
\setlength{\parindent}{0cm}
\setlength{\parskip}{5mm}

% Configurando caption
\captionsetup[table]{skip=4mm}

% Estilo de orientación, tamaño y font family de chapter title
\ChTitleVar{\huge\bfseries\sffamily\raggedleft}

    \newgeometry{top=2cm,bottom=2cm,left=3cm,right=2cm}
\begin{titlepage}
    \begin{minipage}{2.16cm}
        \begin{center}
            \includegraphics[width=2.16cm,height=3.2cm]{img/logo_universidad.png}
        \end{center}
    \end{minipage}
    \hfill
    \begin{minipage}{10cm}
        \begin{center}
            \large{ \textbf{\MakeUppercase{\nombreUniversidad}} }\\
            \normalsize{ \textbf{\MakeUppercase{\nombreFacultad}} }\\
            \small{ \textbf{\MakeUppercase{\nombreCarrera}} }
        \end{center}
    \end{minipage}
    \hfill
    \begin{minipage}{2.55cm}
        \begin{center}
            \includegraphics[width=2.55cm,height=2.55cm]{img/logo_facultad.png}
        \end{center}
    \end{minipage}
    \vspace{5cm}\\

    \begin{center}
        \textbf{\large\MakeUppercase{\nombreProyecto}}
    \end{center}

    \vspace{4cm}
    \begin{center}
        \Large\descripcion
    \end{center}

    \vspace{2cm}
    \large\textnormal{\textbf{Presentado por:} \nombreAutor}\\
    \large\textnormal{\textbf{Tutor:} \nombreTutor}\\

    \vspace{2.5cm}
    \begin{center}
        \textbf{\MakeUppercase{\nombreCiudadPais}}\\
        \fecha
    \end{center}
\end{titlepage}
\restoregeometry
    
    \frontmatter
        \chapter*{Dedicatoria}
% Sin el "*" en el chapter el capítulo aparecerá en el índice
% \chapter{Dedicatoria}

A mi abuelo que me enseñó a no parar de
estudiar y aprender, a mi abuela que me
transmitió su curiosidad, a mi hermano que
me extiende la mano cuando tengo problemas y a 
mi querida madre que me impregnó de su enorme
fuerza de voluntad y paciencia.
        \chapter*{Agradecimientos}
Agradezco a mi familia que me sostiene, impulsa
y ayuda sin importar las adversidades.

Agradezco también a la vida por todas aquellas personas 
que por azares del destino llegué a conocer, con las 
cuales he pasado inolvidables momentos, mañanas de estudio,
tardes de juegos y paseos, noches de cuestionamientos 
filosóficos y charlas sin sentido.
        \tableofcontents
        \listoffigures
        \listoftables
    
    \mainmatter
        \chapter{Recursos LaTeX}

\begin{center}
    \textit{En que consiste este capítulo (opcional).}
\end{center}

Explicación e introducción al capítulo. 

Blah blah blah blah blah blah blah blah blah blah blah
blah blah blah blah blah blah blah blah blah blah blah 
blah blah blah blah blah blah blah blah blah blah blah 
blah blah blah blah blah blah.

Blah blah blah blah blah blah blah blah blah blah blah 
blah blah blah blah blah blah.

\section{Fuente}

\subsection{Familia}
{\ttfamily typewriter (máquina de escribir)}

{\sffamily sans serif}

{\rmfamily roman}

\subsection{Forma}
\textbf{texto en negritas}\\
\textit{texto en itálicas}\\
\textsl{texto inclinado}\\
\texttt{texto en estilo máquina de escribir}\\
\textsc{texto en mayúsculas pequeñas}

\subsection{Tamaño}
{\tiny texto de prueba}\\
{\scriptsize texto de prueba}\\
{\footnotesize texto de prueba}\\
{\small texto de prueba}\\
{\normalsize texto de prueba}\\
{\large texto de prueba}\\
{\Large texto de prueba}\\
{\LARGE texto de prueba}\\
{\huge texto de prueba}\\
{\Huge texto de prueba}

\section{Listado}

\subsection{No numerados}
\begin{itemize}
    \item Item 1
    \item Item 2
    \item Item 3
\end{itemize}

\subsection{Numerados}
\begin{enumerate}
    \item Item 1
    \item Item 2
    \item Item 3
\end{enumerate}


\section{Referenciación con APA}

\subsection{Citación como parte de párrafo}

\subsection*{Un autor}

Como menciona \citeA{libro:ejemplo}, no es la única
forma de citar.

\subsection*{Varios autores}

Como menciona \citeA{libro:ejemplo_varios_autores}, no es la única
forma de citar.

\subsection{Citación en la parte final}

\subsection*{Un autor}

Duis fringilla tristique neque. Sed interdum libero ut metus.
Pellentesque placerat. Nam rutrum augue a leo. Morbi sed 
elit sit amet ante lobortis sollicitudin.Praesent blandit 
blandit mauris citumoris totalis. \cite{libro:ejemplo}.

\subsection*{Varios autores}

Duis fringilla tristique neque. Sed interdum libero ut metus.
Pellentesque placerat. Nam rutrum augue a leo. Morbi sed 
elit sit amet ante lobortis sollicitudin.Praesent blandit 
blandit mauris citumoris totalis. \cite{libro:ejemplo_varios_autores}.

\subsection{Citación con número de página}

Como menciona \citeA[p.~5]{libro:ejemplo}, no es la única
forma de citar.

Duis fringilla tristique neque. Sed interdum libero ut metus.
Pellentesque placerat. Nam rutrum augue a leo. Morbi sed 
elit sit amet ante lobortis sollicitudin.Praesent blandit 
blandit mauris citumoris totalis. 
\cite[p.~7-12]{libro:ejemplo_varios_autores}.

\subsection{Citación anexos}
Duis fringilla tristique neque. Sed interdum libero ut metus.
Pellentesque placerat (Ver Anexo A).

\section{Figuras}
Referenciando a la figura \ref{fig:ejemplo}.

\begin{figure}[H]
    \centering
    \includegraphics[width=8cm]{img/capitulo_1/figura_ejemplo.png}
    \caption{Explicación de la figura (Aquí)}
    Fuente: Adaptada de Apellido, N. (2000) \textit{Nombre del libro}.
    Editorial o universidad que lo publicó.
    \label{fig:ejemplo}
\end{figure}

\section{Tablas}
Sección en la que se detallan el uso de tablas.

\subsection{Corto}
Con respecto a la tabla \ref{tabla:ejemplo}, se tiene la siguiente 
información.

\begin{table}[H]
    \begin{center}
        \caption{Título de la tabla}
        \label{tabla:ejemplo}
        \begin{tabular}{c|c|c|c|}
            \cline{2-4}
            & \textbf{Columna 1} & \textbf{Columna 2} & \textbf{Columna 3} \\ \hline
            \multicolumn{1}{|c|}{\textbf{Fila 1}} & item               & item               & item               \\ \hline
            \multicolumn{1}{|c|}{\textbf{Fila 2}} & item               & item               & item               \\ \hline
            \multicolumn{1}{|c|}{\textbf{Fila 3}} & item               & item               & item               \\ \hline
        \end{tabular}
    \end{center}
    \vspace{4mm}
    Nota. Extraída de Apellido, N. (2000) \textit{Nombre del libro}.
    Editorial o universidad que lo publicó.
\end{table}

\subsection{Multipágina}
Stique neque. Sed interdum libero ut metus.
Pellentesque placerat. Nam rutrum augue a leo. Morbi sed 
elit sit amet ante lobortis sollicitudin.Praesent blandit 
blandit mauris. Praesent lectus tellus, aliquet aliquam,
luctus a, egestas a, turpis.Mauris lacinia lorem sit amet
ipsum. Nunc quis urna dictum turpis accumsan semper. Tabla
\ref{tabla:tabla_largo_ejemplo}.

\begin{longtable}{p{2cm}|p{3cm}|p{3cm}|p{3cm}|p{3cm}|}
    \caption{Titulo de tabla multipágina} \label{tabla:tabla_largo_ejemplo} \\
    \cline{2-5}
    \multicolumn{1}{l|}{} & \multicolumn{1}{c|}{\textbf{Columna 1}} & \multicolumn{1}{c|}{\textbf{Columna 2}} & \multicolumn{1}{c|}{\textbf{Columna 3}} & \multicolumn{1}{c|}{\textbf{Columna 4}}\\ \hline
    \endfirsthead

    \multicolumn{5}{c}{{\tablename{} \thetable{} -- Continuación de tabla previa}} \\
    \multicolumn{5}{l}{} \\
    \cline{2-5}
    \multicolumn{1}{l|}{} & \multicolumn{1}{c|}{\textbf{Columna 1}} & \multicolumn{1}{c|}{\textbf{Columna 2}} & \multicolumn{1}{c|}{\textbf{Columna 3}} & \multicolumn{1}{c|}{\textbf{Columna 4}}\\ \hline
    \endhead

    \multicolumn{5}{c}{{Continua en la siguiente página.}} \\
    \endfoot

    \multicolumn{5}{l}{} \\
    \multicolumn{5}{l}{Nota. Extraída de Apellido, N. (2000) \textit{Nombre del libro}. Editorial o universidad que lo publicó.} \\
    \endlastfoot

    \multicolumn{1}{|c|}{\textbf{Fila 1}} & Lorem ipsum dolor sit amet, consectetuer adipiscing elit. & Lorem ipsum dolor sit amet, consectetuer adipiscing elit. & Lorem ipsum dolor sit amet, consectetuer adipiscing elit. & Lorem ipsum dolor sit amet, consectetuer adipiscing elit. \\ \hline
    \multicolumn{1}{|c|}{\textbf{Fila 2}} & Lorem ipsum dolor sit amet, consectetuer adipiscing elit. & Lorem ipsum dolor sit amet, consectetuer adipiscing elit. & Lorem ipsum dolor sit amet, consectetuer adipiscing elit. & Lorem ipsum dolor sit amet, consectetuer adipiscing elit. \\ \hline
    \multicolumn{1}{|c|}{\textbf{Fila 3}} & Lorem ipsum dolor sit amet, consectetuer adipiscing elit. & Lorem ipsum dolor sit amet, consectetuer adipiscing elit. & Lorem ipsum dolor sit amet, consectetuer adipiscing elit. & Lorem ipsum dolor sit amet, consectetuer adipiscing elit. \\ \hline
    \multicolumn{1}{|c|}{\textbf{Fila 4}} & Lorem ipsum dolor sit amet, consectetuer adipiscing elit. & Lorem ipsum dolor sit amet, consectetuer adipiscing elit. & Lorem ipsum dolor sit amet, consectetuer adipiscing elit. & Lorem ipsum dolor sit amet, consectetuer adipiscing elit. \\ \hline
    \multicolumn{1}{|c|}{\textbf{Fila 5}} & Lorem ipsum dolor sit amet, consectetuer adipiscing elit. & Lorem ipsum dolor sit amet, consectetuer adipiscing elit. & Lorem ipsum dolor sit amet, consectetuer adipiscing elit. & Lorem ipsum dolor sit amet, consectetuer adipiscing elit. \\ \hline
    \multicolumn{1}{|c|}{\textbf{Fila 6}} & Lorem ipsum dolor sit amet, consectetuer adipiscing elit. & Lorem ipsum dolor sit amet, consectetuer adipiscing elit. & Lorem ipsum dolor sit amet, consectetuer adipiscing elit. & Lorem ipsum dolor sit amet, consectetuer adipiscing elit. \\ \hline  
    \multicolumn{1}{|c|}{\textbf{Fila 7}} & Lorem ipsum dolor sit amet, consectetuer adipiscing elit. & Lorem ipsum dolor sit amet, consectetuer adipiscing elit. & Lorem ipsum dolor sit amet, consectetuer adipiscing elit. & Lorem ipsum dolor sit amet, consectetuer adipiscing elit. \\ \hline    
    \multicolumn{1}{|c|}{\textbf{Fila 8}} & Lorem ipsum dolor sit amet, consectetuer adipiscing elit. & Lorem ipsum dolor sit amet, consectetuer adipiscing elit. & Lorem ipsum dolor sit amet, consectetuer adipiscing elit. & Lorem ipsum dolor sit amet, consectetuer adipiscing elit. \\ \hline    
    \multicolumn{1}{|c|}{\textbf{Fila 9}} & Lorem ipsum dolor sit amet, consectetuer adipiscing elit. & Lorem ipsum dolor sit amet, consectetuer adipiscing elit. & Lorem ipsum dolor sit amet, consectetuer adipiscing elit. & Lorem ipsum dolor sit amet, consectetuer adipiscing elit. \\ \hline    
    \multicolumn{1}{|c|}{\textbf{Fila 10}} & Lorem ipsum dolor sit amet, consectetuer adipiscing elit. & Lorem ipsum dolor sit amet, consectetuer adipiscing elit. & Lorem ipsum dolor sit amet, consectetuer adipiscing elit. & Lorem ipsum dolor sit amet, consectetuer adipiscing elit. \\ \hline    
    \multicolumn{1}{|c|}{\textbf{Fila 11}} & Lorem ipsum dolor sit amet, consectetuer adipiscing elit. & Lorem ipsum dolor sit amet, consectetuer adipiscing elit. & Lorem ipsum dolor sit amet, consectetuer adipiscing elit. & Lorem ipsum dolor sit amet, consectetuer adipiscing elit. \\ \hline    
    \multicolumn{1}{|c|}{\textbf{Fila 12}} & Lorem ipsum dolor sit amet, consectetuer adipiscing elit. & Lorem ipsum dolor sit amet, consectetuer adipiscing elit. & Lorem ipsum dolor sit amet, consectetuer adipiscing elit. & Lorem ipsum dolor sit amet, consectetuer adipiscing elit. \\ \hline    
\end{longtable}

\section{Fórmulas matemáticas}

\section*{Simple}
Referenciando fórmula en parrafo, fórmula \ref{eq:euler}.

\begin{equation}
    e^{i\pi} + 1 = 0
    \label{eq:euler}
\end{equation}

\section*{Matrices}
Referenciando fórmula en parrafo, fórmula \ref{eq:matriz}.

\begin{equation}
    \left[
    \begin{matrix}
     a & b & c \\
     d & e & f \\
     g & h & i
    \end{matrix}
    \right]
    \label{eq:matriz}
\end{equation}

\section*{Límites}
Referenciando fórmula en parrafo, fórmula \ref{eq:limites}.

\begin{equation}
    \lim_{x\rightarrow\infty}\frac{3+x}{x^2} 
    \label{eq:limites}
\end{equation}

\section{Diagramas de flujo}
A pesar de que en {\LaTeX} es posible realizar geniales diagramas de flujo,
a la larga es complicado mantenerlos, en ese sentido,
es recomendable utilizar alternativas como \url{https://app.diagrams.net/}
o mas conocido como draw.io, tiene instaladores para todos los sistemas 
operativos. 

\begin{center}
    \begin{tikzpicture}[node distance = 3cm, auto]
        \footnotesize
        % Nodos
        \node [block] (init) {Inicio};
        \node [block, below of=init] (test) {Nodo A};
        \node [block, right of=test] (search) {Nodo B};
        \node [decision, below of=test] (test_done) {Decisión A};
        \node [block, below of=test_done] (intregate) {Nodo C};
        \node [cloud, left of=intregate] (code) {Extra};
        \node [decision, below of=intregate] (all_done) {Decisión B};
        \node [block, below of=all_done] (stop) {Nodo D};
        % Cordenadas
        \coordinate[left of=test] (il);
        \coordinate[left of=all_done] (al);
        % Líneas
        \path [line, dashed] (code) -- (intregate);
        \path [line, dashed] (test_done) -| node [near start] {si} (code);
        \path [line] (init) -- (test);
        \path [line] (test) -- (test_done);
        \path [line] (test_done) -- node {si}(intregate);
        \path [line] (test_done) -| node  [near start] {no} (search);
        \path [line] (intregate) -- (all_done);
        \path [line] (all_done.west) |- node [near end] {no} ([xshift=-2.3cm]al) -- ([xshift=-2.3cm]il)-- (test.west);
        \path [line] (all_done) -- node {si} (stop);
        \path [line] (search) -- (test);
    \end{tikzpicture}
\end{center}

        % importar demás capítulos acá ...

    {
        \backmatter
        \bibliographystyle{apacite}
        \bibliography{bibliografia}
    }

    \appendix
    \begin{appendices}
        \chapter{Título de Anexo A}

Contenido de Anexo A
        \chapter{Título de Anexo B}

Contenido de Anexo B
    \end{appendices}
    % \clearpage
    % \addappheadtotoc
    % \appendixpage
    
\end{document}